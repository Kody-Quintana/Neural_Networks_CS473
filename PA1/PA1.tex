\documentclass[14pt]{article}
\usepackage{../quintana}
\begin{document}
\begin{flushleft}
 
\large
Kody Quintana\\
CS 473\\
Artificial Neural Networks\\
\today\\
\boldmath

\begin{center}
Programming assignment \# 1
\end{center}

\question
\textbf{Given:}
	A system with 10 inputs and 1000 randomly generated instances, $x_i$'s,
	for each input to give a full input set
	\[X = [x_1  \ x_2  \ \cdots \ x_{10}]\]
	where
	\[x_1 = [x_{1,1} \  x_{2,1}  \ \cdots \  x_{1000,1}]\]
	and output labels,
	\[Y = [y_1 \ y_2 \ \cdots \ y_{1000}]\]
\closequestion

\question
\textbf{Find:}
	\begin{enumerate}[label = \textbf{(\alph*)}]
	\item
		The hyper-dimensional linear solution to the system
		utilizing a perceptron model with the above parameters.
	\item
		Solve the system with a sigmoid activation function.
	\item
		Solve with a pseduo-inverse, $X = \text{inv}(A'A)A'b.$
	\end{enumerate}
\closequestion


\question
\textbf{Hint:}
	for part \textbf{(b)} remember to normalize your labels between 0 and 1.
	In fact, it might be wise to one hot encode the label data to values of 0 or 1.
	You can use the rule
	\[Y_{\text{\textbf{normalized}}_{i}} = 0
	\ \text{if} \
	(Y_i \leq 0.5)\]
	and 
	\[Y_{\text{\textbf{normalized}}_{i}} = 1
	\ \text{if} \
	(Y_i \geq 0.5)\]
\closequestion


\cppfile{cpp/main.cpp}

\cppfile{cpp/neuralnet.hpp}

\end{flushleft}
\end{document}
